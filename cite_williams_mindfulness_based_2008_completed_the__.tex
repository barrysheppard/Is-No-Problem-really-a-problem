\cite{williams_mindfulness-based_2008} completed the first study specifically looking at MBCT and bipolar disorder. The sample included 68 participants who had at least one major depressive episode. Of the sample, 17 participants met the criteria for bipolar in remission and had suicidal ideation. Participants were randomly allocated into a group who received immediate MBCT treatment (24 unipolar, 9 bipolar) and a control group (27 unipolar, 8 bipolar) who were on a wait-list for treatment.
The study found that both unipolar and bipolar participants in the MBCT treatment group had significantly reduced levels of depression when compared to the control group. In addition, the study found that bipolar participants in the MBCT group had significantly lower levels of anxiety when compared with bipolar participants in the control group. The study had some limitations. It had a small sample of participants with bipolar and did not look at manic symptoms. Due to sample focusing on suicidal ideation, it may not necessarily generalise its results to bipolar, and the reasons for the higher anxiety levels in the control group are difficult to determine. Despite these limitations, however, it did demonstrate the potential for further research.