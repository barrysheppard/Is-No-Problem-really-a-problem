Bipolar disorder is one of the top 10 most disabling conditions in the world \citep{kupfer_increasing_2005}. It effects an estimated 1 percent of the population in its classic manic-depressive form \citep{weissman_cross-national_1996}, also known as the bipolar I subtype. An additional 2 percent of the population suffer from other subtypes and related disorders \citep{kupfer_increasing_2005}. The pathogenesis of the disorder remains a subject of study, with genetic factors possibly contributing to bipolar risk \citep{fears_multisystem_2014}. In 1995, the estimated cost of the disorder in the US alone was \$45 billion per annum \citep{wyatt_economic_1995}. On average, one-fifth of individuals with bipolar die by suicide \cite{jamison_touched_1996} (Jamison, 1996, p41). It “may account for one-quarter of all completed suicides” \citep[Diagnostic and Statistical Manual of Mental Disorders][p131]{_diagnostic_2013}. Episodes of hypomanic and depressive mood and energy levels characterise bipolar and related disorders.