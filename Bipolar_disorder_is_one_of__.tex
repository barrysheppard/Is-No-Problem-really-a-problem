Bipolar disorder is one of the top 10 most disabling conditions in the world (Kupfer, 2005). It effects an estimated 1% of the population in its classic manic-depressive form (Weissman et al., 1996), also known as the bipolar I subtype. An additional 2% of the population suffer from other subtypes and related disorders (Kupfer, 2005). The pathogenesis of the disorder remains a subject of study, with genetic factors possibly contributing to bipolar risk (Fears, Service, & Kremeyer, 2014). In 1995, the estimated cost of the disorder in the US alone was $45 billion per annum (Wyatt & Henter, 1995). On average, one-fifth of individuals with bipolar die by suicide (Jamison, 1996, p41). It “may account for one-quarter of all completed suicides” (Diagnostic and Statistical Manual of Mental Disorders, 2013, p131).
Episodes of hypomanic and depressive mood and energy levels characterise bipolar and related disorders. 

Bipolar I disorder is a “modern understanding of the classic manic-depressive” (Diagnostic and Statistical Manual of Mental Disorders, 2013, p123). Manic episodes are periods of over a week in which there is an abnormally elevated mood and energy levels that cause impairments in social and work life. These often include a decreased need for sleep, and sometimes feelings of euphoria. Hypomania is the milder occurrence of mania, lasting at least four consecutive days and not necessarily causing life impairments. Depressive episodes are periods of two weeks in which a depressed mood or loss of interest occurs. These periods also include the possibility of weight loss, insomnia, fatigue, feelings of worthlessness, diminished ability to think, or thoughts of death.



The primary treatment for bipolar is mood stabilisers, often lithium, although recently anti-convulsants have proved successful to flatten the extreme emotions. Anti-psychotic medication is used to control manic episodes. Anti-depressant medication is used to control the depressive episodes; however anti-depressants can trigger and increase manic episodes (Zhang et al., 2013). In extreme cases, electroconvulsive therapy is the preferred option (Dierckx, Heijnen, van den Broek, & Birkenhäger, 2012). A combination of medication and psychotherapy, usually, has the best results. 
Mindfulness, in short, is an awareness of, and a presence in, the now. It derives from the Buddhist meditation practice, involving a focus on the breath, the body, and sensations. In so doing, the practitioner practices self-regulation. In maintaining a focus of attention on the now, they develop a conscious awareness of their thoughts and thought patterns. The practice proposes that there are two distinct kinds of thinking; self-reference in the now and self-reference across time, a theory supported by fMRI studies (Farb et al., 2007). 

Bipolar disorder is one of the top 10 most disabling conditions in the world (Kupfer, 2005). It effects an estimated 1% of the population in its classic manic-depressive form (Weissman et al., 1996), also known as the bipolar I subtype. An additional 2% of the population suffer from other subtypes and related disorders (Kupfer, 2005). The pathogenesis of the disorder remains a subject of study, with genetic factors possibly contributing to bipolar risk (Fears, Service, & Kremeyer, 2014). In 1995, the estimated cost of the disorder in the US alone was $45 billion per annum (Wyatt & Henter, 1995). On average, one-fifth of individuals with bipolar die by suicide (Jamison, 1996, p41). It “may account for one-quarter of all completed suicides” (Diagnostic and Statistical Manual of Mental Disorders, 2013, p131).
Episodes of hypomanic and depressive mood and energy levels characterise bipolar and related disorders. Bipolar I disorder is a “modern understanding of the classic manic-depressive” (Diagnostic and Statistical Manual of Mental Disorders, 2013, p123). Manic episodes are periods of over a week in which there is an abnormally elevated mood and energy levels that cause impairments in social and work life. These often include a decreased need for sleep, and sometimes feelings of euphoria. Hypomania is the milder occurrence of mania, lasting at least four consecutive days and not necessarily causing life impairments. Depressive episodes are periods of two weeks in which a depressed mood or loss of interest occurs. These periods also include the possibility of weight loss, insomnia, fatigue, feelings of worthlessness, diminished ability to think, or thoughts of death.
The primary treatment for bipolar is mood stabilisers, often lithium, although recently anti-convulsants have proved successful to flatten the extreme emotions. Anti-psychotic medication is used to control manic episodes. Anti-depressant medication is used to control the depressive episodes; however anti-depressants can trigger and increase manic episodes (Zhang et al., 2013). In extreme cases, electroconvulsive therapy is the preferred option (Dierckx, Heijnen, van den Broek, & Birkenhäger, 2012). A combination of medication and psychotherapy, usually, has the best results. 
Mindfulness, in short, is an awareness of, and a presence in, the now. It derives from the Buddhist meditation practice, involving a focus on the breath, the body, and sensations. In so doing, the practitioner practices self-regulation. In maintaining a focus of attention on the now, they develop a conscious awareness of their thoughts and thought patterns. The practice proposes that there are two distinct kinds of thinking; self-reference in the now and self-reference across time, a theory supported by fMRI studies (Farb et al., 2007). 

 In modern life, it is typical for individuals to think of one thing while doing another. Individuals anticipate lunch while shopping, worry about work when watching television, and live for the weekend from Monday to Friday. The mind automatically evaluates and assesses the steps taken, or to be taken, to bridge the gap between then and now. This cognitive approach is an evolutionary one which is advantageous in many goal orientated situations. In today’s world, with its many abstract problems, always using this problem-solving thinking can lead to frustration and anxiety. In individuals who score low on mindfulness, the amygdala, the part of the brain involved in the fight or flight response, is on high alert even when resting (Way, David, Eisenberger, & Lieberman, 2010).


One negative thought, when dwelled upon, activates other negative thoughts until eventually a constellation of worry and anxiety is present. “Brooding is the problem, not the solution” (Williams & Penman, 2011, p30). Mindfulness, with its focus on, and acceptance of, the now provides a different approach. It allows an individual “to experience the world — calmly and non-judgmentally” (Williams & Penman, 2011, p45). With increased mindfulness, an individual gradually becomes aware “your thoughts are not you” (Williams & Penman, 2011, p64). 

Jon Kabat-Zinn introduced mindfulness into the field of psychology, taking lessons from Buddhist meditation and developing a non-secular program. He began his Mindfulness-based stress reduction program (MBSR) in 1979 and has since produced many papers on the benefits of mindfulness (Kabat-Zinn, 1982; Kabat-Zinn, Lipworth, & Burney, 1985; Kabat-Zinn et al., 1992; Miller, Fletcher, & Kabat-Zinn, 1995; Kabat-Zinn et al., 1998; Samuelson, Carmody, Kabat-Zinn, & Bratt, 2007; Ludwig & Kabat-Zinn, 2008). Researchers developed scales to measure how mindful an individual is; the Mindful Attention Awareness Scale (Brown & Ryan, 2003), the Kentucky Inventory of Mindfulness Skills (Baer, Smith, & Allen, 2004), the Toronto Mindfulness Scale (Lau et al., 2006), the Cognitive and Affective Mindfulness Scale-Revised (Feldman, Hayes, Kumar, Greeson, & Laurenceau, 2007), and the Philadelphia mindfulness scale (Cardaciotto, Herbert, Forman, Moitra, & Farrow, 2008). Using these scales research investigated correlates with mindfulness.

A meta-analysis of 39 studies relating to mindfulness (Hofmann, Sawyer, Witt, & Oh, 2010) found that a mindfulness-based therapy was effective in improving anxiety and mood symptoms. Empirically supported benefits of mindfulness, as collated by Davis & Hayes (2011), suggests mindfulness positively correlates with reduced rumination, reduced stress, improved working memory, improved focus, reduced emotional reactivity, increased cognitive flexibility, and increased relationship satisfaction among other benefits.

One specific program of therapy is the mindfulness-based cognitive therapy (MBCT). It is an eight week program of meditation developed by Zindel Segal, Mark Williams, and John Teasdale adapted from the mindfulness-based stress reduction (MBSR) program of Jon Kabat-Zinn. While MBCT is similar in many ways to MBSR, they differ in some regards. MBSR aims to increase resilience to stress of individuals going through difficult circumstances. MBSR instructors can come from a number of backgrounds. In contrast, MBCT is a mental health treatment designed to prevent relapse in individuals who have suffered from depression. Only licensed health providers qualify to be MBCT instructors. The program teaches people to be “aware of their thoughts without judgment; viewing negative (positive and neutral) thoughts as passing mental events rather than as facts” (Williams et al., 2008).

As a specific clinical treatment for depression, MBCT is of particular interest in the treatment of bipolar disorder. As anti-depressive drugs stimulate the release or inhibit the uptake of neurotransmitters in the brain, there is an associated risk of increasing the effect and recurrence of manic episodes. This side effect makes alternative forms of treatment for depressive symptoms desirable. As a treatment offered by licensed health professionals, it provides a consistent approach for comparison across studies. To date, a small number of studies have looked at MBCT as a treatment for bipolar. Six studies are detailed below. In all cases, participants during these studies continued with their ongoing treatment. Treatment may have included a course of medication, which could differ between individuals, and confound results to some degree. 

Williams et al. (2008) completed the first study specifically looking at MBCT and bipolar disorder. The sample included 68 participants who had at least one major depressive episode. Of the sample, 17 participants met the criteria for bipolar in remission and had suicidal ideation. Participants were randomly allocated into a group who received immediate MBCT treatment (24 unipolar, 9 bipolar) and a control group (27 unipolar, 8 bipolar) who were on a wait-list for treatment.

The study found that both unipolar and bipolar participants in the MBCT treatment group had significantly reduced levels of depression when compared to the control group. In addition, the study found that bipolar participants in the MBCT group had significantly lower levels of anxiety when compared with bipolar participants in the control group. The study had some limitations. It had a small sample of participants with bipolar and did not look at manic symptoms. Due to sample focusing on suicidal ideation, it may not necessarily generalise its results to bipolar, and the reasons for the higher anxiety levels in the control group are difficult to determine. Despite these limitations, however, it did demonstrate the potential for further research.

 Expanding upon the Williams et al. (2008) study Miklowitz et al. (2009) completed another pilot study which looked at a sample of 16 participants with bipolar in remission who completed an eight week MBCT program. Researchers compared self-reported scales before and after the program. The study found reductions in depressive symptoms. Researchers found manic symptoms reduced slightly, which is notable as an increase in manic symptoms would be a potential concern with the reduction in depressive symptoms. Anxiety decreased modestly, however without a control group to compare with, it is difficult to draw conclusions. This study does have some limitations. The sample size is small and included no control group. As an exploratory study the paper succeeded on building on the Williams et al. (2008) study and is the first investigation of the effect of MBCT on manic symptoms.

 In 2010, Weber et al. completed a feasibility trial on MBCT’s long-term effects on bipolar disorder. Fifteen participants with bipolar attended at least four MBCT sessions. Participants completed self-reported questionnaires before the program, 1 month later, and 2 months after that. The study did not find a significant improvement in mindfulness scores on the KIMS (Baer, Smith, & Allen, 2004), depressive, or hypomania scales across time. The study did find that the KIMS score significantly negatively correlated with depression severity prior to the MBCT program.
 This study did not report significant benefits for participants with bipolar, either in symptoms or mindfulness skills. The authors speculate that the KIMS score may not measure the same aspects of mindfulness focused on by the MBCT. Again, the study does have certain limitations; the sample size is small and included no control group. The study draws a link between symptoms and mindfulness scores as determined by the KIMS scale. In addition, three months following the MBCT course, 67% percent of participant’s self-reported a moderate or greater benefit from the program which raises questions of whether a placebo effect was involved.
 In contrast to the previous studies, Howells, Ives-Deliperi, Horn, & Stein (2012) used EEG research to study brain wave activity. The study had 12 participants with bipolar and 9 control participants, before and after an 8 week MBCT program. Prior to MBCT, participants with bipolar showed greater brain activity “over the frontal and cingulate cortex during resting” when compared with control participants (Howells et al., 2012). This difference was not apparent during the attentional task. This activity may indicate decreased attentional readiness. MBCT improved this readiness slightly. In addition, prior to MBCT participants with bipolar “showed activation of non-relevant information processing over the frontal cortex” (Howells et al., 2012). Following MBCT, this reduced non-relevant information processing. This study is consistent with previous research which dealt more generally with meditation. Areas of the brain associated with “attention, interoception and sensory processing” the prefrontal cortex and right anterior insula were denser in participants who meditated than those who did not (Lazar et al., 2005).

Ives-Deliperi, Howells, Stein, Meintjes, & Horn (2013) completed an additional neuro-imaging study using fMRI to study the effects of MBCT on participants with bipolar. The sample included 23 patients with bipolar disorder and 10 control subjects. Of the patients with bipolar disorder, 16 had MBCT training while the remaining 7 were assigned to a control group. Self-reported questionnaires rated mindfulness, emotion regulation, anxiety, and symptoms of stress. Participants performed a resting and a meditative mindfulness exercise during which researchers measured their brain activity with fMRI. Researchers recorded results before and after the MBCT training.

 For the self-reported measures, significant differences were found between the MBCT and wait groups, with the MBCT group higher in mindfulness and emotional regulation, and lower in anxiety. Depression scores did improve, but not significantly. The MBCT group also had significantly higher performance than the wait group on tasks measuring working memory, spatial memory, and verbal fluency. The fMRI scans showed significant signal increases in the medial prefrontal cortex and posterior cingulate cortex in the MBCT group, following training, when compared to the wait group. Participants with bipolar also displayed significant signal decreases in the medial prefrontal cortex when compared to the control group, a key area for the disorder (Anand, Li, Wang, Lowe, & Dzemidzic, 2009). The data from this study suggests MBCT “improves mindfulness and emotion regulation and reduces anxiety in bipolar disorder” as seen by increased activation in the medial prefrontal cortex (Ives-Deliperi et al., 2013).
The sixth, and final study, was a randomised controlled trial with a larger sample (Perich, Manicavasagar, Mitchell, Ball, & Hadzi-Pavlovic, 2013). A sample of participants with bipolar were randomised into a MBCT group of 38 and a “treatment as usual” control group of 29. Self-report questionnaires rated time to recurrence of depressive and manic episodes, the number of recurrences, a depressive rating, a mania rating, and anxiety scales. The MBCT group completed the eight week MBCT course, and both groups were reassessed 12 months after. The results found no significant reduction in time to recurrence, number of recurrences, depressive rating, or mania rating. This study did not support the generalisation of previous findings on MBCT and depression to MBCT and depressive symptoms in bipolar disorder. The study did, however, find “some evidence for an effect on anxiety symptoms” (Perich et al., 2013). 

From the 6 studies, only 2 supported the idea that MBCT reduced depressive symptoms while 3 of the 6 studies were unable to find significant differences. The 2 studies which supported the idea involved a total of 25 bipolar participants, compared to a total of 76 for the 3 studies which were unable to find differences. Based on this limited data set, MBCT is not as effective in dealing with depressive symptoms in bipolar individuals as it is with unipolar individuals.

Of the 6 studies, only 3 looked at manic symptoms of which 1 study supported a slight reduction in symptoms while 2 were unable to find a significant change. The study which did show a reduction had a sample of 16 participants with bipolar compared with a total sample of 43 for the 2 studies not supporting. Based on this limited data set, it would appear that MBCT is not effective in dealing with manic symptoms in bipolar individuals. Both of the neuro-imaging studies found changes in frontal cortex activity following MBCT which resulted in brain activity closer to that of an individual without bipolar.

Five of the studies supported the idea that MBCT reduces levels of anxiety co-morbid with bipolar disorder. While this effect does not directly relate to the disorder itself, the effects of a co-morbid anxiety disorder on the prognosis for bipolar is notable. Anxiety disorders are co-morbid in approximately three-fourths of individuals with bipolar disorder (Diagnostic and Statistical Manual of Mental Disorders, 2013, p132). Those individuals who do have a co-mobid anxiety disorder are more likely to relapse, have worse sleep disturbance, are more likely to require medication, and have an increased risk of suicide (Hawke, Provencher, Parikh, & Zagorski, 2013).
 In summary, research in this particular field is limited, both in number of studies and sample size. Given the nature of the disorder both in duration and treatment, various potential confounding variables remain uncontrolled. Findings so far are positive for MBCT as a treatment for co-morbid anxiety, but not for the characteristic symptoms of mania and depression with bipolar disorder. This raises some interesting questions on how exactly depression in unipolar and bipolar cases differ and whether the effects of MBCT on brain activity as seen with the neuro-imaging research, can give insight. Future studies would benefit from a control group which, rather than waiting, participated in a non-mindfulness meditative therapy as in Jain et al. (2007), to account for the placebo effect. Despite its lack of effect on bipolar symptoms, MBCT appears to be of some benefit due to its effects on anxiety levels. As such, its continued use with patients with bipolar disorder is beneficial and would give the opportunity to collect additional data and expand the understanding of both the disorder and the therapy itself.
  