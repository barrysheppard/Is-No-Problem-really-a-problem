Of the 6 studies, only 3 looked at manic symptoms of which 1 study supported a slight reduction in symptoms while 2 were unable to find a significant change. The study which did show a reduction had a sample of 16 participants with bipolar compared with a total sample of 43 for the 2 studies not supporting. Based on this limited data set, it would appear that MBCT is not effective in dealing with manic symptoms in bipolar individuals. Both of the neuro-imaging studies found changes in frontal cortex activity following MBCT which resulted in brain activity closer to that of an individual without bipolar.