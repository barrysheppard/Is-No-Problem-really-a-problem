\cite{ives-deliperi_effects_2013} completed an additional neuro-imaging study using fMRI to study the effects of MBCT on participants with bipolar. The sample included 23 patients with bipolar disorder and 10 control subjects. Of the patients with bipolar disorder, 16 had MBCT training while the remaining 7 were assigned to a control group. Self-reported questionnaires rated mindfulness, emotion regulation, anxiety, and symptoms of stress. Participants performed a resting and a meditative mindfulness exercise during which researchers measured their brain activity with fMRI. Researchers recorded results before and after the MBCT training.
For the self-reported measures, significant differences were found between the MBCT and wait groups, with the MBCT group higher in mindfulness and emotional regulation, and lower in anxiety. Depression scores did improve, but not significantly. The MBCT group also had significantly higher performance than the wait group on tasks measuring working memory, spatial memory, and verbal fluency. The fMRI scans showed significant signal increases in the medial prefrontal cortex and posterior cingulate cortex in the MBCT group, following training, when compared to the wait group. Participants with bipolar also displayed significant signal decreases in the medial prefrontal cortex when compared to the control group, a key area for the disorder \cite{anand_resting_2009}. The data from this study suggests MBCT “improves mindfulness and emotion regulation and reduces anxiety in bipolar disorder” as seen by increased activation in the medial prefrontal cortex \cite{ives-deliperi_effects_2013}.
  