In summary, research in this particular field is limited, both in number of studies and sample size. Given the nature of the disorder both in duration and treatment, various potential confounding variables remain uncontrolled. Findings so far are positive for MBCT as a treatment for co-morbid anxiety, but not for the characteristic symptoms of mania and depression with bipolar disorder. This raises some interesting questions on how exactly depression in unipolar and bipolar cases differ and whether the effects of MBCT on brain activity as seen with the neuro-imaging research, can give insight. Future studies would benefit from a control group which, rather than waiting, participated in a non-mindfulness meditative therapy as in \citet{jain_randomized_2007}, to account for the placebo effect. Despite its lack of effect on bipolar symptoms, MBCT appears to be of some benefit due to its effects on anxiety levels. As such, its continued use with patients with bipolar disorder is beneficial and would give the opportunity to collect additional data and expand the understanding of both the disorder and the therapy itself.
  
  