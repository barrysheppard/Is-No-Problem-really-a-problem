One specific program of therapy is the mindfulness-based cognitive therapy (MBCT). It is an eight week program of meditation developed by Zindel Segal, Mark Williams, and John Teasdale adapted from the mindfulness-based stress reduction (MBSR) program of Jon Kabat-Zinn. While MBCT is similar in many ways to MBSR, they differ in some regards. MBSR aims to increase resilience to stress of individuals going through difficult circumstances. MBSR instructors can come from a number of backgrounds. In contrast, MBCT is a mental health treatment designed to prevent relapse in individuals who have suffered from depression. Only licensed health providers qualify to be MBCT instructors. The program teaches people to be “aware of their thoughts without judgment; viewing negative (positive and neutral) thoughts as passing mental events rather than as facts” \cite{williams_mindfulness-based_2008}.
  