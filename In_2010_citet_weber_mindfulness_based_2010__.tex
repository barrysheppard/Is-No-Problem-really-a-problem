In 2010, \citet{weber_mindfulness-based_2010} completed a feasibility trial on MBCT’s long-term effects on bipolar disorder. Fifteen participants with bipolar attended at least four MBCT sessions. Participants completed self-reported questionnaires before the program, 1 month later, and 2 months after that. The study did not find a significant improvement in mindfulness scores on the KIMS \cite{baer_assessment_2004}, depressive, or hypomania scales across time. The study did find that the KIMS score significantly negatively correlated with depression severity prior to the MBCT program. This study did not report significant benefits for participants with bipolar, either in symptoms or mindfulness skills. The authors speculate that the KIMS score may not measure the same aspects of mindfulness focused on by the MBCT. Again, the study does have certain limitations; the sample size is small and included no control group. The study draws a link between symptoms and mindfulness scores as determined by the KIMS scale. In addition, three months following the MBCT course, 67\% percent of participant’s self-reported a moderate or greater benefit from the program which raises questions of whether a placebo effect was involved.
  