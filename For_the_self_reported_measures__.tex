For the self-reported measures, significant differences were found between the MBCT and wait groups, with the MBCT group higher in mindfulness and emotional regulation, and lower in anxiety. Depression scores did improve, but not significantly. The MBCT group also had significantly higher performance than the wait group on tasks measuring working memory, spatial memory, and verbal fluency. The fMRI scans showed significant signal increases in the medial prefrontal cortex and posterior cingulate cortex in the MBCT group, following training, when compared to the wait group. Participants with bipolar also displayed significant signal decreases in the medial prefrontal cortex when compared to the control group, a key area for the disorder \cite{anand_resting_2009}. The data from this study suggests MBCT “improves mindfulness and emotion regulation and reduces anxiety in bipolar disorder” as seen by increased activation in the medial prefrontal cortex \cite{ives-deliperi_effects_2013}.
The sixth, and final study, was a randomised controlled trial with a larger sample \cite{perich_randomized_2013}. A sample of participants with bipolar were randomised into a MBCT group of 38 and a “treatment as usual” control group of 29. Self-report questionnaires rated time to recurrence of depressive and manic episodes, the number of recurrences, a depressive rating, a mania rating, and anxiety scales. The MBCT group completed the eight week MBCT course, and both groups were reassessed 12 months after. The results found no significant reduction in time to recurrence, number of recurrences, depressive rating, or mania rating. This study did not support the generalisation of previous findings on MBCT and depression to MBCT and depressive symptoms in bipolar disorder. The study did, however, find “some evidence for an effect on anxiety symptoms” \cite{perich_randomized_2013}.