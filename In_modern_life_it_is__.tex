In modern life, it is typical for individuals to think of one thing while doing another. Individuals anticipate lunch while shopping, worry about work when watching television, and live for the weekend from Monday to Friday. The mind automatically evaluates and assesses the steps taken, or to be taken, to bridge the gap between then and now. This cognitive approach is an evolutionary one which is advantageous in many goal orientated situations. In today’s world, with its many abstract problems, always using this problem-solving thinking can lead to frustration and anxiety. In individuals who score low on mindfulness, the amygdala, the part of the brain involved in the fight or flight response, is on high alert even when resting \citep{way_dispositional_2010}. 
One negative thought, when dwelled upon, activates other negative thoughts until eventually a constellation of worry and anxiety is present. “Brooding is the problem, not the solution” \citep[][p30]{williams_mindfulness:_2011}. Mindfulness, with its focus on, and acceptance of, the now provides a different approach. It allows an individual “to experience the world — calmly and non-judgmentally” \citep[][p45]{williams_mindfulness:_2011}. With increased mindfulness, an individual gradually becomes aware “your thoughts are not you” \citep[][p64]{williams_mindfulness:_2011}.
  
  