From the 6 studies, only 2 supported the idea that MBCT reduced depressive symptoms while 3 of the 6 studies were unable to find significant differences. The 2 studies which supported the idea involved a total of 25 bipolar participants, compared to a total of 76 for the 3 studies which were unable to find differences. Based on this limited data set, MBCT is not as effective in dealing with depressive symptoms in bipolar individuals as it is with unipolar individuals.
Of the 6 studies, only 3 looked at manic symptoms of which 1 study supported a slight reduction in symptoms while 2 were unable to find a significant change. The study which did show a reduction had a sample of 16 participants with bipolar compared with a total sample of 43 for the 2 studies not supporting. Based on this limited data set, it would appear that MBCT is not effective in dealing with manic symptoms in bipolar individuals. Both of the neuro-imaging studies found changes in frontal cortex activity following MBCT which resulted in brain activity closer to that of an individual without bipolar.
Five of the studies supported the idea that MBCT reduces levels of anxiety co-morbid with bipolar disorder. While this effect does not directly relate to the disorder itself, the effects of a co-morbid anxiety disorder on the prognosis for bipolar is notable. Anxiety disorders are co-morbid in approximately three-fourths of individuals with bipolar disorder \citep[Diagnostic and Statistical Manual of Mental Disorders][p132]{_diagnostic_2013}. Those individuals who do have a co-mobid anxiety disorder are more likely to relapse, have worse sleep disturbance, are more likely to require medication, and have an increased risk of suicide \citep{hawke_comorbid_2013}.
  