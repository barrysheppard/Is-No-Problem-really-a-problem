Bipolar I disorder is a “modern understanding of the classic manic-depressive” (Diagnostic and Statistical Manual of Mental Disorders, 2013, p123). Manic episodes are periods of over a week in which there is an abnormally elevated mood and energy levels that cause impairments in social and work life. These often include a decreased need for sleep, and sometimes feelings of euphoria. Hypomania is the milder occurrence of mania, lasting at least four consecutive days and not necessarily causing life impairments. Depressive episodes are periods of two weeks in which a depressed mood or loss of interest occurs. These periods also include the possibility of weight loss, insomnia, fatigue, feelings of worthlessness, diminished ability to think, or thoughts of death.