In contrast to the previous studies, Howells, Ives-Deliperi, Horn, & Stein (2012) used EEG research to study brain wave activity. The study had 12 participants with bipolar and 9 control participants, before and after an 8 week MBCT program. Prior to MBCT, participants with bipolar showed greater brain activity “over the frontal and cingulate cortex during resting” when compared with control participants (Howells et al., 2012). This difference was not apparent during the attentional task. This activity may indicate decreased attentional readiness. MBCT improved this readiness slightly. In addition, prior to MBCT participants with bipolar “showed activation of non-relevant information processing over the frontal cortex” (Howells et al., 2012). Following MBCT, this reduced non-relevant information processing. This study is consistent with previous research which dealt more generally with meditation. Areas of the brain associated with “attention, interoception and sensory processing” the prefrontal cortex and right anterior insula were denser in participants who meditated than those who did not (Lazar et al., 2005).
  