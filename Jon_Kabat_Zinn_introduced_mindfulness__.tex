Jon Kabat-Zinn introduced mindfulness into the field of psychology, taking lessons from Buddhist meditation and developing a non-secular program. He began his Mindfulness-based stress reduction program (MBSR) in 1979 and has since produced many papers on the benefits of mindfulness \citep{kabat-zinn_outpatient_1982,kabat-zinn_clinical_1985,kabat-zinn_effectiveness_1992,miller_three-year_1995,kabat-zinn_influence_1998,samuelson_mindfulness-based_2007,ludwig_mindfulness_2008}. Researchers developed scales to measure how mindful an individual is; the Mindful Attention Awareness Scale (Brown & Ryan, 2003), the Kentucky Inventory of Mindfulness Skills (Baer, Smith, & Allen, 2004), the Toronto Mindfulness Scale (Lau et al., 2006), the Cognitive and Affective Mindfulness Scale-Revised (Feldman, Hayes, Kumar, Greeson, & Laurenceau, 2007), and the Philadelphia mindfulness scale (Cardaciotto, Herbert, Forman, Moitra, & Farrow, 2008). Using these scales research investigated correlates with mindfulness.