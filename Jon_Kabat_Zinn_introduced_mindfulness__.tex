Jon Kabat-Zinn introduced mindfulness into the field of psychology, taking lessons from Buddhist meditation and developing a non-secular program. He began his Mindfulness-based stress reduction program (MBSR) in 1979 and has since produced many papers on the benefits of mindfulness (Kabat-Zinn, 1982; Kabat-Zinn, Lipworth, & Burney, 1985; Kabat-Zinn et al., 1992; Miller, Fletcher, & Kabat-Zinn, 1995; Kabat-Zinn et al., 1998; Samuelson, Carmody, Kabat-Zinn, & Bratt, 2007; Ludwig & Kabat-Zinn, 2008). Researchers developed scales to measure how mindful an individual is; the Mindful Attention Awareness Scale (Brown & Ryan, 2003), the Kentucky Inventory of Mindfulness Skills (Baer, Smith, & Allen, 2004), the Toronto Mindfulness Scale (Lau et al., 2006), the Cognitive and Affective Mindfulness Scale-Revised (Feldman, Hayes, Kumar, Greeson, & Laurenceau, 2007), and the Philadelphia mindfulness scale (Cardaciotto, Herbert, Forman, Moitra, & Farrow, 2008). Using these scales research investigated correlates with mindfulness.